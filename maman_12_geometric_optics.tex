\documentclass[12pt]{article}
\usepackage{fontspec}
\usepackage{polyglossia} % use instead of babel
\setdefaultlanguage{hebrew}
\setotherlanguage{english}

\newfontfamily\hebrewfont[Script=Hebrew]{David CLM}


\begin{document}

\begin{center}
    {\LARGE \textbf{ממ"ן 12}}\\
    {\textbf{אופטיקה גיאומטרית}}
\end{center}

\begin{itemize}
    \item מגישים: תומר רוזנלפלד, שי רואימי
    \item תאריך ביצוע הניסוי: 19 יולי 2025
    \item מדריך הניסוי: ד"ר סילביו ריינהורן
\end{itemize}

\section*{מטרות הניסוי:}
\begin{enumerate}
    \item מדידת מקדם השבירה בפרספקס
    \item מדידת מרחק מוקד בעדשה מרכזת ועדשה מפזרת
\end{enumerate}

\section*{רקע תיאורטי:}
\subsection*{חוק סנל}

במעבר אור מתווך חומר אחד לאחר, זווית השבירה תלויה במקדם השבירה של שני החומרים, ובזווית הפגיעה.
חוק סנל מתאר את הקשר בין זוויות השבירה והכניסה של האור בחומרים שונים.
מקדם שבירה מוגדר בנוסחא:

\begin{equation}
    n = \frac{c}{v}
\end{equation}

כאשר $c$ היא מהירות האור בריק ו-$v$ היא מהירות האור בחומר.


חוק סנל קובע את הקשר בין זוויות השבירה והכניסה של האור בחומרים שונים:

\begin{equation}
n_1 \sin(\theta_1) = n_2 \sin(\theta_2)
\end{equation}

כאשר $n_1$ ו-$n_2$ הם מקדמי השבירה של החומרים 1 ו-2, ו-$\theta_1$ ו-$\theta_2$ הן זוויות הפגיעה והשבירה בהתאמה.

\subsection*{עדשות}
עדשות הן אובייקטים אופטיים המיועדים לשנות את כיוון קרני האור העוברים דרכם, בניסויים נעזר בשני סוגי עדשות: מרכזת ומפזרת.

מרחק המוקד של עדשה הוא המרחק בין העדשה לנקודה בה מתרכזות קרני האור העוברים דרכה.
מרחק המוקד של עדשה מרכזת הוא חיובי, ואילו מרחק המוקד של עדשה מפזרת הוא שלילי.

נוסחאת גאוס היא נוסחא המתארת את הקשר בין מרחק המוקד של העדשה, המרחק בין העדשה לאובייקט והמרחק בין העדשה לתמונה המתקבלת:
\begin{equation}
\frac{1}{f} = \frac{1}{u} + \frac{1}{v}
\end{equation}
כאשר $f$ הוא מרחק המוקד, $u$ הוא המרחק בין העדשה לאובייקט ו-$v$ הוא המרחק בין העדשה לתמונה המתקבלת.

\section*{ניסוי 1 - חוק סנל}
\subsection*{מטרת הניסוי}
מציאת מקדם של פרספקס

\subsection*{רקע תיאורטי}
במעבר אור מתווך חומר אחד לאחר, זווית השבירה תלויה במקדם השבירה של שני החומרים, ובזווית הפגיעה.
חוק סנל מתאר את הקשר בין זוויות השבירה והכניסה של האור בחומרים שונים.
מקדם שבירה מוגדר בנוסחא:

\begin{equation}
    n = \frac{c}{v}
\end{equation}

כאשר $c$ היא מהירות האור בריק ו-$v$ היא מהירות האור בחומר.


חוק סנל קובע את הקשר בין זוויות השבירה והכניסה של האור בחומרים שונים:

\begin{equation}
n_1 \sin(\theta_1) = n_2 \sin(\theta_2)
\end{equation}

כאשר $n_1$ ו-$n_2$ הם מקדמי השבירה של החומרים 1 ו-2, ו-$\theta_1$ ו-$\theta_2$ הן זוויות הפגיעה והשבירה בהתאמה.

\subsection*{מערכת המדידה ומהלך הניסוי}

כיוונו קרן אור כלפי חצי דסקה עשויה פרספקס, ובעזרת מד זווית מדדנו את זוויות השבירה בהנתן זוויות פגיעה שונות. בעזרת ידע זה מצאנו את מקדם השבירה של הפרספקס.

\subsection*{תוצאות וניתוח}

\end{document}